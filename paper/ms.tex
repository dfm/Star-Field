\documentclass[letterpaper, 11pt]{article}
\usepackage{authblk}
\usepackage{graphicx} 
\usepackage{natbib}
\usepackage{amsmath}
\usepackage[left=3cm,top=3cm,right=3cm]{geometry}

\title{More than one Author with different Affiliations}

\renewcommand{\topfraction}{0.85} \renewcommand{\textfraction}{0.1} 
\parindent=0cm

\title{Probabilistic Catalogues from Crowded Star Fields}
\author[1,2]{Brendon J. Brewer}
\author[3]{David W. Hogg}
\author[3]{Daniel Foreman-Mackey}
\affil[1]{UCSB}
\affil[2]{Auckland}
\affil[3]{NYU}

\date{{\bf Title and author order subject to change} \\
{\bf Also nag me if you prefer American English}}

\begin{document}
\maketitle

\section{Introduction}

\section{The Unknown Parameters}
There are $N$ stars in the field, where $N$ is unknown. Each star has an unknown
position $(x,y)$ in the plane of the sky, and an unknown flux $f$. We also
parameterise the {\it distribution} of positions by parameters $\alpha$,
and we parameterise the distribution of fluxes (aka the {\it luminosity
function}) by parameters $\beta$. In summary, the unknown parameters are:
\begin{eqnarray}
\theta = \left\{N, \alpha, \beta, \left\{x_i, y_i\right\}_{i=1}^N, 
\left\{f_i\right\}_{i=1}^N\right\}
\end{eqnarray}

The prior probability distribution for the unknown parameters can be factorised
using the product rule. With a variety of independence assumptions, the prior
can be written as:
\begin{eqnarray}
p(\theta) = p(N)p(\alpha)p(\beta)\prod_{i=1}^N p(x_i, y_i | \alpha)
p(f_i | \beta) 
\end{eqnarray}

Here, we have assumed that the luminosity function does not depend on position.
We have also assumed independence of $N$, $\alpha$ and $\beta$, which means that
learning the value of one would not tell us anything about the values of the
others. Finally, the positions and fluxes of the stars come independently from
a common distribution. If we knew the spatial distribution and the luminosity
function of the stars, then the location and flux of a particular star would not
tell us anything about the location and flux of another star.

\section{Particular Models}
A model for the spatial distribution corresponds to a choice of functional forms
for the distributions $p(x_i, y_i | \alpha)$ and $p(f_i | \beta)$.

\section{A More Realistic Model}
\citep{2008ApJ...682..874K}

\appendix
\section{Pixel Convolved PSF}
Consider a ``true image'' $f_0(x, y)$ with infinite resolution. We now model how
this image gives rise to the observed data. First, it is convolved with a
PSF $g(x, y)$ (assumed normalised to 1) to give the blurred image:
\begin{eqnarray}
f_{\rm{blurred}}(x, y) &=& \iint f_0(x, y)g(x - \delta_x, y - \delta_y)
\, d\delta_x \, d\delta_y
\end{eqnarray}
Then, we observe the blurred image with a certain pixellation, such that
the true flux $F$ within a particular pixel is:
\begin{eqnarray}
F_{\rm{pixel}} &=& \iint_{\rm{pixel}} f_{\rm blurred}(x, y) \,dx\,dy
\end{eqnarray}



\section{Proposal Distributions}



\begin{thebibliography}{99}
\bibitem[Kelly et al.(2008)]{2008ApJ...682..874K} Kelly, B.~C., Fan, X., 
\& Vestergaard, M.\ 2008, ApJ, 682, 874 
\end{thebibliography}

\end{document}

