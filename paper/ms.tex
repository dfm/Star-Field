\documentclass[letterpaper, 11pt]{article}
\usepackage{graphicx} 
\usepackage{natbib}
\usepackage[left=3cm,top=3cm,right=3cm]{geometry}

\renewcommand{\topfraction}{0.85} \renewcommand{\textfraction}{0.1} 
\parindent=0cm

\title{Crowded Star Fields \\ \bf{Obviously need a better title}} 
\author{Brendon J. Brewer, David W. Hogg, Daniel Foreman-Mackey \\ 
\bf{Order subject to change}}

\begin{document} \maketitle

\section{Introduction}

\section{The Unknown Parameters}
The stars are at positions $\{(x_i, 
y_i)\}_{i=1}^n$ and have fluxes $\{f_i\}$. We also include 
hyperparameters, denoted collectively, by $\alpha$, describing the 
distribution of the positions and fluxes of the stars.

\begin{eqnarray} 
p(\{(x_i, y_i, f_i)\} | \alpha) &=&
\prod_{i=1}^n p(x_i, y_i, f_i | \alpha) \\
&=& \prod_{i=1}^n p(x_i, y_i | \alpha)p(f_i | \alpha) 
\end{eqnarray}
Here, we have assumed that the ``luminosity function'' 
(distribution of fluxes) does not depend on position.

In summary, the unknown parameters are $\{(x_i, y_i, f_i)\}_{i=1}^n$
and $\alpha$.

\section{The Luminosity Function}
\citep{2008ApJ...682..874K}

\appendix
\section{Pixel Convolved PSF}
Consider a ``true image'' $f(x, y)$. We now model how this image
gives rise to the observed data. First, it is convolved with a
PSF $g(x,y)$ to give the 

\section{Reversible Jump Moves}



\begin{thebibliography}{99}
\bibitem[Kelly et al.(2008)]{2008ApJ...682..874K} Kelly, B.~C., Fan, X., 
\& Vestergaard, M.\ 2008, ApJ, 682, 874 
\end{thebibliography}

\end{document}

